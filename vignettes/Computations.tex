% Options for packages loaded elsewhere
\PassOptionsToPackage{unicode}{hyperref}
\PassOptionsToPackage{hyphens}{url}
\PassOptionsToPackage{dvipsnames,svgnames,x11names}{xcolor}
%
\documentclass[
  letterpaper,
  DIV=11,
  numbers=noendperiod]{scrartcl}

\usepackage{amsmath,amssymb}
\usepackage{iftex}
\ifPDFTeX
  \usepackage[T1]{fontenc}
  \usepackage[utf8]{inputenc}
  \usepackage{textcomp} % provide euro and other symbols
\else % if luatex or xetex
  \usepackage{unicode-math}
  \defaultfontfeatures{Scale=MatchLowercase}
  \defaultfontfeatures[\rmfamily]{Ligatures=TeX,Scale=1}
\fi
\usepackage{lmodern}
\ifPDFTeX\else  
    % xetex/luatex font selection
\fi
% Use upquote if available, for straight quotes in verbatim environments
\IfFileExists{upquote.sty}{\usepackage{upquote}}{}
\IfFileExists{microtype.sty}{% use microtype if available
  \usepackage[]{microtype}
  \UseMicrotypeSet[protrusion]{basicmath} % disable protrusion for tt fonts
}{}
\makeatletter
\@ifundefined{KOMAClassName}{% if non-KOMA class
  \IfFileExists{parskip.sty}{%
    \usepackage{parskip}
  }{% else
    \setlength{\parindent}{0pt}
    \setlength{\parskip}{6pt plus 2pt minus 1pt}}
}{% if KOMA class
  \KOMAoptions{parskip=half}}
\makeatother
\usepackage{xcolor}
\setlength{\emergencystretch}{3em} % prevent overfull lines
\setcounter{secnumdepth}{-\maxdimen} % remove section numbering
% Make \paragraph and \subparagraph free-standing
\ifx\paragraph\undefined\else
  \let\oldparagraph\paragraph
  \renewcommand{\paragraph}[1]{\oldparagraph{#1}\mbox{}}
\fi
\ifx\subparagraph\undefined\else
  \let\oldsubparagraph\subparagraph
  \renewcommand{\subparagraph}[1]{\oldsubparagraph{#1}\mbox{}}
\fi


\providecommand{\tightlist}{%
  \setlength{\itemsep}{0pt}\setlength{\parskip}{0pt}}\usepackage{longtable,booktabs,array}
\usepackage{calc} % for calculating minipage widths
% Correct order of tables after \paragraph or \subparagraph
\usepackage{etoolbox}
\makeatletter
\patchcmd\longtable{\par}{\if@noskipsec\mbox{}\fi\par}{}{}
\makeatother
% Allow footnotes in longtable head/foot
\IfFileExists{footnotehyper.sty}{\usepackage{footnotehyper}}{\usepackage{footnote}}
\makesavenoteenv{longtable}
\usepackage{graphicx}
\makeatletter
\def\maxwidth{\ifdim\Gin@nat@width>\linewidth\linewidth\else\Gin@nat@width\fi}
\def\maxheight{\ifdim\Gin@nat@height>\textheight\textheight\else\Gin@nat@height\fi}
\makeatother
% Scale images if necessary, so that they will not overflow the page
% margins by default, and it is still possible to overwrite the defaults
% using explicit options in \includegraphics[width, height, ...]{}
\setkeys{Gin}{width=\maxwidth,height=\maxheight,keepaspectratio}
% Set default figure placement to htbp
\makeatletter
\def\fps@figure{htbp}
\makeatother

\KOMAoption{captions}{tableheading}
\makeatletter
\makeatother
\makeatletter
\makeatother
\makeatletter
\@ifpackageloaded{caption}{}{\usepackage{caption}}
\AtBeginDocument{%
\ifdefined\contentsname
  \renewcommand*\contentsname{Table of contents}
\else
  \newcommand\contentsname{Table of contents}
\fi
\ifdefined\listfigurename
  \renewcommand*\listfigurename{List of Figures}
\else
  \newcommand\listfigurename{List of Figures}
\fi
\ifdefined\listtablename
  \renewcommand*\listtablename{List of Tables}
\else
  \newcommand\listtablename{List of Tables}
\fi
\ifdefined\figurename
  \renewcommand*\figurename{Figure}
\else
  \newcommand\figurename{Figure}
\fi
\ifdefined\tablename
  \renewcommand*\tablename{Table}
\else
  \newcommand\tablename{Table}
\fi
}
\@ifpackageloaded{float}{}{\usepackage{float}}
\floatstyle{ruled}
\@ifundefined{c@chapter}{\newfloat{codelisting}{h}{lop}}{\newfloat{codelisting}{h}{lop}[chapter]}
\floatname{codelisting}{Listing}
\newcommand*\listoflistings{\listof{codelisting}{List of Listings}}
\makeatother
\makeatletter
\@ifpackageloaded{caption}{}{\usepackage{caption}}
\@ifpackageloaded{subcaption}{}{\usepackage{subcaption}}
\makeatother
\makeatletter
\@ifpackageloaded{tcolorbox}{}{\usepackage[skins,breakable]{tcolorbox}}
\makeatother
\makeatletter
\@ifundefined{shadecolor}{\definecolor{shadecolor}{rgb}{.97, .97, .97}}
\makeatother
\makeatletter
\makeatother
\makeatletter
\makeatother
\ifLuaTeX
  \usepackage{selnolig}  % disable illegal ligatures
\fi
\IfFileExists{bookmark.sty}{\usepackage{bookmark}}{\usepackage{hyperref}}
\IfFileExists{xurl.sty}{\usepackage{xurl}}{} % add URL line breaks if available
\urlstyle{same} % disable monospaced font for URLs
\hypersetup{
  pdftitle={Details of computations in the package icpack},
  pdfauthor={Paul Eilers},
  colorlinks=true,
  linkcolor={blue},
  filecolor={Maroon},
  citecolor={Blue},
  urlcolor={Blue},
  pdfcreator={LaTeX via pandoc}}

\title{Details of computations in the package \texttt{icpack}}
\author{Paul Eilers}
\date{2024-04-01}

\begin{document}
\maketitle
\ifdefined\Shaded\renewenvironment{Shaded}{\begin{tcolorbox}[borderline west={3pt}{0pt}{shadecolor}, breakable, interior hidden, boxrule=0pt, enhanced, frame hidden, sharp corners]}{\end{tcolorbox}}\fi

\hypertarget{introduction}{%
\subsection{Introduction}\label{introduction}}

The package \texttt{icpack} fits survival models with a smooth
semi-parametric baseline to interval-censored data, using P-splines. The
proper amount of smoothing is obtained using a mixed model algorithm.
The core is an Expectation-Maximization (EM) algorithm with proper
calculation of standard errors. We have paid special attention to
efficient computation, exploiting high-level matrix operations.

This document presents technical details about the M-step that maximizes
the penalized likelihood. Straightforward organization of the
computations leads to very large matrices. That can be avoided by
recognizing and expliting their very regular structures.

The notation is a somewhat experimental mix of mathematics and
\texttt{R} code.

\hypertarget{simple-poisson-regression}{%
\subsubsection{Simple Poisson
regression}\label{simple-poisson-regression}}

Our approach is based on Poisson regression. In the simple case without
interval censoring, the data are the exact times of occurrence of either
an event or censoring. We divide the time axis into a relatively large
number, say \texttt{nt\ =\ 100}, intervals, which we call bins to avoid
confusion with censoring intervals. In each bin we count the number of
subjects at risk and the number of events, giving the vectors \texttt{r}
and \texttt{y}, each with \texttt{nt} elements.

Each subjects adds a 1 to all bins in \texttt{r} in which it is at risk.
If it experiences the event in bin \texttt{j}, 1 is added to
\texttt{y{[}j{]}}; if it is censored it does not contribute anything to
\texttt{y}.

We indicate the hazard with the vector \texttt{h}. If \texttt{mu} is the
vector of expected values of \texttt{y}, then \texttt{mu\ =\ h\ *\ r} .
We do not model \texttt{h} itself, but its logarithm, \texttt{eta},
writing it as a sum of B-splines: \texttt{eta\ =\ B\ \%*\%\ cb}, where
the columns of \texttt{B} contain the B-splines and \texttt{cb} holds
their coefficients.

To get a smooth log-hazard, we use a roughness penalty, which will be
introduced later on.

The Poisson log-likelihood is \texttt{ll\ =\ sum(...)} and maximizing it
with respect to \texttt{cb} leads to the non-linear system of equations
\texttt{t(B)\ \%*\%\ (y\ -\ mu)\ =\ 0}. If \texttt{cb\_} is an
approximation to the solution, \texttt{mu\_\ =\ exp(B\ \%*\%\ cb\_)},
and \texttt{M\_\ =\ diag(mu\_)}, iterative solution of

\texttt{(t(B)\ \%*\%\ M\_\ \%*\%\ B)\ \%*\%\ cb\ =\ t(B)\ \%*\%\ (y\ -\ mu\_\ +\ M\_\ \%*\%\ B\ \%*\%\ cb\_)}

leads to the final solution.

The roughness penalty is
\texttt{lambda\ *\ sum((D\ \%*\%\ cb)\ \^{}\ 2)}, with
\texttt{D\ =\ diff(diag(nb),\ diff\ =\ 2)}, a sum of squared (second
order) differences of adjacent elements of \texttt{cb}. It was advocated
by Eilers and Marx, as the penalty for P-splines. The system of
equations changes to

\texttt{(t(B)\ \%*\%\ M\_\ \%*\%\ B\ +\ lambda\ *\ t(D)\ \%*\%\ D)\ \%*\%\ cb\ =\ t(B)\ \%*\%\ (y\ -\ mu\_\ +\ M\_\ \%*\%\ B\ \%*\%\ cb\_)}.

The computing load is light. The number of B-splines, \texttt{nb}, does
not have to be large, say 20, so \texttt{B} is a relatively small matrix
of 100 by 20, independent of the number of subjects in the data. As we
will see soon, covariates complicate the picture a lot.

\hypertarget{poisson-regression-with-covariates}{%
\subsubsection{Poisson regression with
covariates}\label{poisson-regression-with-covariates}}

To handle covariates, given in the matrix \texttt{X} with \texttt{n}
rows and \texttt{nx} columns, we assume proportional hazards and a
smooth baseline hazard vector \texttt{h0\ =\ exp(B\ \%*\%\ cb)}. With
\texttt{H{[}i,\ j{]}}, the hazard for subject \texttt{i} in bin
\texttt{j}, we have
\texttt{H{[}i,\ {]}\ =\ exp(X{[}i,\ {]}\ \%*\%\ cx\ +\ B\ \%*\%\ cb)}.

We can no longer work with vectors. The data are coded in two matrices,
\texttt{R} and \texttt{Y}. Row \texttt{i} of \texttt{R} tells us in
which bins subject \texttt{i} was at risk, while row \texttt{i} of
\texttt{Y} records when the event occurred for this subject, if it
occurred. Hence \texttt{Y{[}i,\ j{]}\ =\ 1} in case of an event of
subject \texttt{i} in bin \texttt{j}. If subject \texttt{ì} was
censored, then row \texttt{ì} of \texttt{Y} contains only zeros.

To do straightforward Poisson regression, we must vectorize \texttt{R}
and \texttt{Y} and construct an appropriate design matrix from
\texttt{B} and \texttt{X}. Assume that we vectorize by placing the rows
of \texttt{R} and \texttt{Y} in vectors and joining them. to form
\texttt{r} and \texttt{y}. Note that \texttt{r} and \texttt{y} are now
different from their definitions in the case without covariates.

We must repeat the basis matrix \texttt{B} for each subject and row of
\texttt{X} for each bin. It is convenient to use Kronecker products:
\texttt{BB\ =\ kronecker(rep(1,\ n),\ B)} and
\texttt{XX\ =\ kronecker(X,\ rep(1,\ nt))}. Then
\texttt{Q\ =\ cbind(BB,\ XX)}gives us the proper design matrix for
penalized Poisson regression. The size of \texttt{BB} is
\texttt{n\ *\ nt} rows and \texttt{nb} columns, while \texttt{XX} has
the same number of rows and \texttt{nx} columns. With 1000 subjects, 100
bins, 10 B-splines and 10 covariates, \texttt{Q} has 100000 rows and 20
columns, or 2 million elements.

The core of Poisson regression is the computation of
\texttt{M\ =\ t(Q)\ \%*\%\ W\ \%*\%\ Q}, where
\texttt{W\ =\ diag(mu\_)}. That can be achieved without forming
\texttt{Q} explicitly. We can partition \texttt{M} as

\texttt{M\ =\ rbind(cbind(Mbb,\ Mbx),\ cbind(t(Mbx),\ Mxx)))},

with

\texttt{Mbb\ =\ t(BB)\ \%*\%\ W\ \%*\%\ BB},
\texttt{Mbx\ =\ t(BB)\ \%*\%\ W\ \%*\%\ XX} and
\texttt{Mxx\ =\ t(XX)\ \%*\%\ W\ \%*\%\ XX}. These expressions show how
the component matrices could be computed in an inefficient way by
forming \texttt{BB} and \texttt{XX} and their weighted inner products.
We apply efficient shortcuts instead.

Because of the Kronecker structure of \texttt{BB} and \texttt{XX} we
find that, with the expected counts in the matrix \texttt{Mu}, we have
that \texttt{Mu{[}i,\ j{]}\ =\ R{[}i,\ j{]}\ *\ exp(Eta{[}i,\ j{]})},
with \texttt{Eta{[}i,\ j{]}\ =\ eta0{[}i{]}\ +\ f{[}j{]}}, and
\texttt{eta0\ =\ c(B\ \%*\%\ cb)} and \texttt{f\ =\ c(X\ \%*\%\ cx)}. It
is necessary to use \texttt{c(.)} in both computations, because the
product of a matrix and a vector returns a matrix with one column, not a
vector. The \texttt{outer} function of R computes the sums conveniently:
\texttt{Eta\ =\ outer(eta0,\ f,\ \textquotesingle{}+\textquotesingle{})},
but if we do not apply \texttt{c(.)}, the result is a four-dimensional
array, not a matrix.

Because \texttt{W\ =\ diag(mu\_)} is a diagonal matrix, we compute
\texttt{W\ \%*\%\ BB} as \texttt{mu\_\ *\ BB}, eliminating the
construction of \texttt{W} and a matrix product. As
\texttt{BB\ =\ kronecker(rep(1,\ n),\ B)}, it is \texttt{B} repeated
\texttt{n} times and stacked. Consider two terms in
\texttt{t(BB)\ \%*\%\ (mu\_\ *\ BB)} :

\texttt{mu\_{[}1{]}\ *\ t(BB{[}1,{]})\ \%*\%\ BB{[}1,\ {]}\ +\ mu\_{[}n\ +\ 1{]}\ *\ t(BB{[}n\ +\ 1,\ {]})\ \%*\%\ BB{[}n\ +\ 1,\ {]}}.

Because \texttt{BB{[}n\ +\ 1,\ {]}\ =\ BB{[}1,\ {]}\ =\ B{[}1,\ {]}}, we
can write it as

\texttt{(mu\_{[}1{]}\ +\ mu\_{[}n\ +\ 1{]})\ *\ t(B{[}1,{]})\ \%*\%\ B{[}1,\ {]}},
showing that we can first add the proper elements of \texttt{mu\_} and
then multiply with the outer product of the first row of \texttt{B} with
itself. The same reasoning applies to rows 2, \texttt{n\ +\ 2}, etc. The
sums are in fact the sums of the rows of \texttt{Mu} and thus
\texttt{Mbb\ =\ t(B)\ \%*\%\ (wb\ *\ B)} with
\texttt{wb\ =\ rowSums(Mu)}.

A similar reasoning give us that
\texttt{Mxx\ =\ t(X)\ \%*\%\ (wx\ *\ X)} with
\texttt{wx\ =\ colsums(Mu)}, and the weighted cross product of
\texttt{B} and \texttt{X} is obtained as
\texttt{Mbx\ =\ t(B)\ \%*\%\ Mu\ \%*\%\ X}.



\end{document}
